%!TEX program = xelatex
\documentclass[landscape, oneside, a4paper, cs4size]{book}

\def\marginset#1#2{                      % 页边设置 \marginset{left}{top}
\setlength{\oddsidemargin}{#1}         % 左边(书内侧)装订预留空白距离
\iffalse                   % 如果考虑左侧(书内侧)的边注区则改为\iftrue
\reversemarginpar
\addtolength{\oddsidemargin}{\marginparsep}
\addtolength{\oddsidemargin}{\marginparwidth}
\fi
\setlength{\evensidemargin}{0mm}       % 置0
\iffalse                   % 如果考虑右侧(书外侧)的边注区则改为\iftrue
\addtolength{\evensidemargin}{\marginparsep}
\addtolength{\evensidemargin}{\marginparwidth}
\fi
% \paperwidth = h + \oddsidemargin+\textwidth+\evensidemargin + h
\setlength{\hoffset}{\paperwidth}
\addtolength{\hoffset}{-\oddsidemargin}
\addtolength{\hoffset}{-\textwidth}
\addtolength{\hoffset}{-\evensidemargin}
\setlength{\hoffset}{0.5\hoffset}
\addtolength{\hoffset}{-1in}           % h = \hoffset + 1in
%\setlength{\voffset}{-1in}             % 0 = \voffset + 1in
\setlength{\topmargin}{\paperheight}
\addtolength{\topmargin}{-\headheight}
\addtolength{\topmargin}{-\headsep}
\addtolength{\topmargin}{-\textheight}
\addtolength{\topmargin}{-\footskip}
\addtolength{\topmargin}{#2}           % 上边预留装订空白距离
\setlength{\topmargin}{0.5\topmargin}
}
% 调整页边空白使内容居中,两参数分别为纸的左边和上边预留装订空白距离
\marginset{125mm}{200mm}


%\usepackage{ctex}
\usepackage{bm}
%\usepackage[fleqn]{amsmath}
\usepackage{harpoon}
\usepackage{fontspec}
\usepackage{listings}
\usepackage[left=1cm,right=1cm,top=1.2cm,bottom=1cm,columnsep=1cm,dvipdfm]{geometry}
\usepackage{setspace}
\usepackage{bm}
\usepackage{cmap}
\usepackage{cite}
\usepackage{float}
\usepackage{xeCJK}
\usepackage{amsthm}
\usepackage{amsmath}
\usepackage{amssymb}
\usepackage{multirow}
\usepackage{multicol}
\usepackage{setspace}
\usepackage{enumerate}
\usepackage{indentfirst}
\usepackage{adjmulticol}
\usepackage{titlesec}
\usepackage{xcolor,minted}
\usepackage{xeCJK}
\allowdisplaybreaks
%\setlength{\parindent}{0em}
%\setlength{\mathindent}{0pt}
\lstset{breaklines}
\let\cleardoublepage\relax
\titleformat{\chapter}{\normalfont\normalsize\sffamily}{\thechapter}{10pt}{}
\titleformat{\section}{\normalfont\footnotesize\sffamily}{\thesection}{1em}{}
\titleformat{\subsection}{\normalfont\footnotesize\sffamily}{\thesubsection}{1em}{}
\titleformat{\subsubsection}{\normalfont\footnotesize\sffamily}{\thesubsubsection}{1em}{}
\titlespacing*{\chapter} {0pt}{5pt}{5pt}
\titlespacing*{\section} {0pt}{0pt}{0pt}
\titlespacing*{\subsection} {0pt}{0pt}{0pt}
\titlespacing*{\subsubsection}{0pt}{0pt}{0pt}
%configure fonts
\setmonofont{Ubuntu Mono}[Scale=0.8]
%\setmonofont{FiraCode-Retina}[Scale=0.8]
%\setCJKmainfont{FandolSong-Regular}
%\setCJKsansfont{SourceHanSans-Medium}
%\setCJKmonofont[Scale=0.8]{STXihei}
\usepackage{yfonts}

\usepackage{fancyhdr}

\renewcommand{\theFancyVerbLine}{\sffamily \textcolor[rgb]{0.5,0.5,0.5}{\scriptsize {\arabic{FancyVerbLine}}}}

\usemintedstyle{tango}

\setminted[cpp]{
	style=xcode,
	mathescape,
	linenos,
	autogobble,
	baselinestretch=0.8,
	tabsize=2,
	fontsize=\normalsize,
	%bgcolor=Gray,
	frame=single,
	framesep=1mm,
	framerule=0.3pt,
	numbersep=1mm,
	breaklines=true,
	breaksymbolsepleft=2pt,
	%breaksymbolleft=\raisebox{0.8ex}{ \small\reflectbox{\carriagereturn}}, %not moe!
	%breaksymbolright=\small\carriagereturn,
	breakbytoken=false,
}
\setminted[java]{
	style=xcode,
	mathescape,
	linenos,
	autogobble,
	baselinestretch=0.8,
	tabsize=2,
	fontsize=\normalsize,
	%bgcolor=Gray,
	frame=single,
	framesep=1mm,
	framerule=0.3pt,
	numbersep=1mm,
	breaklines=true,
	breaksymbolsepleft=2pt,
	%breaksymbolleft=\raisebox{0.8ex}{ \small\reflectbox{\carriagereturn}}, %not moe!
	%breaksymbolright=\small\carriagereturn,
	breakbytoken=false,
}
\setminted[text]{
	style=xcode,
	mathescape,
	linenos,
	autogobble,
	baselinestretch=0.8,
	tabsize=4,
	fontsize=\normalsize,
	%bgcolor=Gray,
	frame=single,
	framesep=1mm,
	framerule=0.3pt,
	numbersep=1mm,
	breaklines=true,
	breaksymbolsepleft=2pt,
	%breaksymbolleft=\raisebox{0.8ex}{ \small\reflectbox{\carriagereturn}}, %not moe!
	%breaksymbolright=\small\carriagereturn,
	breakbytoken=false,
}

\usepackage{lastpage}
\pagestyle{fancy}
\fancypagestyle{plain}{}
\fancyhf{}
\lhead{Shanghai Jiao Tong University × Arondight}
\chead{\leftmark}
\rhead{\thepage/\pageref{LastPage}}
\setlength{\headsep}{1pt}

\usepackage{tocloft}
\makeatletter
\renewcommand{\@cftmaketoctitle}{}
\makeatother

\usepackage{punk}

\begin{document}\scriptsize
	\renewcommand{\thefootnote}{\fnsymbol{footnote}}
	\title{\Huge{{\punkfamily Arondight's Standard Code Library}}\thanks{https://www.github.com/footoredo/Arondight}}
	\author{\emph{Shanghai Jiao Tong University}}
	\date{Dated: \today}
	\maketitle
	\clearpage
	\begin{multicols}{2}
		\tableofcontents
		\clearpage
		\begin{spacing}{0.8}
			\def \source {../source}
\chapter{计算几何}

\inputminted{cpp}{\source/computational-geometry/2d/basis.cpp}
\section{凸包}
\inputminted{cpp}{\source/computational-geometry/2d/convex.cpp}
\section{三角形的心}
\inputminted{cpp}{\source/computational-geometry/2d/triangle.cpp}
\section{半平面交}
\inputminted{cpp}{\source/computational-geometry/2d/half-plane-intersection.cpp}
\section{圆交面积及重心}
\inputminted{cpp}{\source/computational-geometry/2d/circles-intersections.cpp}

\section{三维向量绕轴旋转}
\inputminted{cpp}{\source/computational-geometry/3d/basis.cpp}
\section{三维凸包}
\inputminted{cpp}{\source/computational-geometry/3d/convex.cpp}

\chapter{数论}
\section{$O(m^2\log n)$求线性递推数列第n项}
Given $a_0, a_1, \ldots, a_{m - 1}$\\
	$a_n = c_0 \times a_{n - m} + \cdots + c_{m - 1} \times a_{n - 1}$\\
	Solve for $a_n = v_0 \times a_0 + v_1 \times a_1 + \cdots + v_{m - 1} \times a_{m - 1}$\\
\inputminted{cpp}{\source/number-theory/linear-recurrence.cpp}
\section{求逆元}
\inputminted{cpp}{\source/number-theory/get-inversion.cpp}
\section{中国剩余定理}
\inputminted{cpp}{\source/number-theory/chinese-remainder-theorem.cpp}
\section{魔法CRT}
\inputminted{cpp}{\source/number-theory/magic-crt.cpp}
\section{素性测试}
\inputminted{cpp}{\source/number-theory/primality-test.cpp}
\section{质因数分解}
\inputminted{cpp}{\source/number-theory/pollards-rho-algorithm.cpp}
\section{线下整点}
\inputminted{cpp}{\source/number-theory/integer-lattice-under-segment.cpp}
\section{原根相关}
	\begin{enumerate}
		\item 模$m$有原根的充要条件:$m = 2, 4, p^a, 2p^a$,其中$p$是奇素数;
		\item 求任意数$p$原根的方法:对$\phi(p)$因式分解,即$\phi(p) = p_1^{r_1}p_2^{r_2}\cdots p_k^{r_k}$,若恒成立:
			\[g^{\frac{p - 1}{g}} \neq 1 \pmod{p}\]
				那么$g$就是$p$的原根。
		\item 若模$m$有原根,那么它一共有$\Phi(\Phi(m))$个原根。
	\end{enumerate}

\chapter{代数}
\section{$O(n^2\log n)$求线性递推数列第n项}
Given $a_0, a_1, \cdots , a_{m-1} \\
\indent a_n = c_0 * a_{n-m} + \cdots + c_{m-1} * a_0 \\
\indent a_0 \ is \ the \ nth \ element, \cdots, a_{m-1} \ is \ the \ n+m-1th \ element
$
\inputminted{cpp}{\source/algebra/linear-recursion.cpp}
\section{闪电数论变换与魔力CRT}
\inputminted{cpp}{\source/algebra/NTT+CRT.cpp}
\section{多项式求逆}
Given polynomial a and n, b is the polynomial such that $a * b \equiv 1 (\mod x^n) $
\inputminted{cpp}{\source/algebra/polynomial-inverse.cpp}
\section{多项式除法}
d is quotient and r is remainder
\inputminted{cpp}{\source/algebra/polynomial-divide.cpp}
\section{多项式取指数取对数}
Given polynomial a and n, b is the polynomial such that $b \equiv e^a (\mod x^n)$ or $b \equiv \ln a (\mod x^n)$
\inputminted{cpp}{\source/algebra/polynomial-expandln.cpp}
\chapter{字符串}
\section{后缀数组}
\inputminted{cpp}{\source/string/suffix-array.cpp}
\section{后缀自动机}
\inputminted{cpp}{\source/string/suffix-automaton.cpp}
\section{EX后缀自动机}
\inputminted{cpp}{\source/string/ex-suffix-automaton.cpp}
\section{后缀树}
\begin{enumerate}
	\item 边上的字符区间是左闭右开区间;
	\item 如果要建立关于多个串的后缀树,请用不同的分隔符,并且对于每个叶子结点,去掉和它父亲的连边上出现的第一个分隔符之后的所有字符;
\end{enumerate}
\section{回文自动机}
\inputminted{cpp}{\source/string/palindromic-tree.cpp}

\chapter{数据结构}
\section{KD-Tree}
\inputminted{cpp}{\source/data-structure/kd-tree.cpp}
\section{Treap}
\inputminted{cpp}{\source/data-structure/treap.cpp}
\section{Link/cut Tree}
\inputminted{cpp}{\source/data-structure/link-cut-tree.cpp}
\section{树状数组查询第k小元素}
\inputminted{cpp}{\source/data-structure/kth-element-on-fenwick-tree.cpp}

\chapter{图论}
\section{基础}
\inputminted{cpp}{\source/graph-theory/basis.cpp}
\section{KM}
\inputminted{cpp}{\source/graph-theory/KM.cpp}
\section{点双连通分量}
\texttt{bcc.forest} is a set of connected tree whose vertices are chequered with cut-vertex and BCC.
\inputminted{cpp}{\source/graph-theory/biconnected-graph-vertex.cpp}
\section{边双连通分量}
\inputminted{cpp}{\source/graph-theory/biconnected-graph-edge.cpp}
\section{最小树形图}
\inputminted{cpp}{\source/graph-theory/optimum-branching.cpp}
\section{带花树}
\inputminted{cpp}{\source/graph-theory/blossom-algorithm.cpp}
\section{Dominator Tree}
\inputminted{cpp}{\source/graph-theory/dominator-tree.cpp}
\section{无向图最小割}
\inputminted{cpp}{\source/graph-theory/stoer-wagner-algorithm.cpp}
\section{重口味费用流}
\inputminted{cpp}{\source/graph-theory/zkw-cost-flow.cpp}
\section{2-SAT}
\inputminted{cpp}{\source/graph-theory/2-satisfiability.cpp}

\chapter{其他}
\section{Dancing Links}
\inputminted{cpp}{\source/others/dancing-links.cpp}
\section{蔡勒公式}
0 for Sunday. Day and month is 1-based.
\inputminted{cpp}{\source/others/zellers-congruence.cpp}
%\section{五边形数定理}
%the number of partitions of n:
%$p(n) = \sum_{k \in \mathbb{Z} \backslash \{0\}} (-1)^{k - 1}p(n - \frac{k(3k-1)}{2})$

\chapter{技巧}
\section{真正的释放STL容器内存空间}
\inputminted{cpp}{\source/tricks/truly-release-container-space.cpp}
\section{无敌的大整数相乘取模}
Time complexity $O(1)$.
\inputminted{cpp}{\source/tricks/O1-multiply-mod.cpp}
\section{无敌的读入优化}
\inputminted{cpp}{\source/tricks/unbeatable-input-acceleration.cpp}
\section{梅森旋转算法}
High quality pseudorandom number generator, twice as efficient as rand() with \texttt{-O2}.
C++11 required.
\inputminted{cpp}{\source/tricks/mersenne-twister.cpp}

\chapter{技巧}
\section{无敌的读入优化}
\inputminted{cpp}{\source/hints/input-acceleration.cpp}
\section{真正释放STL内存}
\inputminted{cpp}{\source/hints/STL-memory-release.cpp}

		\end{spacing}
	\end{multicols}
\end{document}
