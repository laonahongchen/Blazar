\documentclass[titlepage,landscape,a4paper,10pt]{article}
\usepackage{listings, color, fontspec, minted, setspace, titlesec, fancyhdr, dingbat, mdframed, multicol}
\usepackage{graphicx, amssymb, amsmath, textcomp, booktabs}
\usepackage[Chinese]{ucharclasses}
\usepackage[left=1.5cm, right=0.7cm, top=1.7cm, bottom=0.0cm]{geometry}

%configure the top corners
\pagestyle{fancy}
\setlength{\headsep}{0.1cm}
\rhead{Page \thepage}
\lhead{上海交通大学 Shanghai Jiao Tong University}

%configure space between the two columns
\setlength{\columnsep}{30pt}

%configure fonts
\setmonofont{Isotype}[Scale=0.8]
\newfontfamily\substitutefont{SimHei}[Scale=0.8]
\setTransitionsForChinese{\begingroup\substitutefont}{\endgroup}

%configure minted to display codes 
\definecolor{Gray}{rgb}{0.9,0.9,0.9}

%remove leading numbers in table of contents
\setcounter{secnumdepth}{0}

%configure section style
%\titleformat{\section}
%	{\normalfont\normalsize}	% The style of the section title
%	{}					% a prefix
%	{0pt}				% How much space exists between the prefix and the title
%	{\quad}				% How the section is represented
\titleformat{\section}{\large}{}{0pt}{}
\titlespacing{\section}{0pt}{0pt}{0pt}

%enable section to start new page automatically
%\let\stdsection\section
%\renewcommand\section{\penalty-100\vfilneg\stdsection}

%\renewcommand\theFancyVerbLine{\arabic{FancyVerbLine}}
\renewcommand{\theFancyVerbLine}{\sffamily \textcolor[rgb]{0.5,0.5,0.5}{\scriptsize {\arabic{FancyVerbLine}}}}

\setminted[cpp]{
	style=xcode,
	mathescape,
	linenos,
	autogobble,
	baselinestretch=0.9,
	tabsize=2,
	fontsize=\normalsize,
	%bgcolor=Gray,
	frame=single,
	framesep=1mm,
	framerule=0.3pt,
	numbersep=1mm,
	breaklines=true,
	breaksymbolsepleft=2pt,
	%breaksymbolleft=\raisebox{0.8ex}{ \small\reflectbox{\carriagereturn}}, %not moe!
	%breaksymbolright=\small\carriagereturn,
	breakbytoken=false,
}
\setminted[java]{
	style=xcode,
	mathescape,
	linenos,
	autogobble,
	baselinestretch=1.0,
	tabsize=2,
	%bgcolor=Gray,
	frame=single,
	framesep=1mm,
	framerule=0.3pt,
	numbersep=1mm,
	breaklines=true,
	breaksymbolsepleft=2pt,
	%breaksymbolleft=\raisebox{0.8ex}{ \small\reflectbox{\carriagereturn}}, %not moe!
	%breaksymbolright=\small\carriagereturn,
	breakbytoken=false,
}
\setminted[text]{
	style=xcode,
	mathescape,
	linenos,
	autogobble,
	baselinestretch=1.0,
	tabsize=2,
	%bgcolor=Gray,
	frame=single,
	framesep=1mm,
	framerule=0.3pt,
	numbersep=1mm,
	breaklines=true,
	breaksymbolsepleft=2pt,
	%breaksymbolleft=\raisebox{0.8ex}{ \small\reflectbox{\carriagereturn}}, %not moe!
	%breaksymbolright=\small\carriagereturn,
	breakbytoken=false,
}

%configure titles
\title{\LARGE{Dreadnought} \\[2ex] \Large{Standard Code Library} }
\date{\today}

%THE SCL BEGINS
\begin{document}
\maketitle

\begin{multicols*}{2}

\begin{spacing}{0}
	\tableofcontents
\end{spacing}
\end{multicols*}

\begin{multicols}{2}

\newpage
\begin{spacing}{0.8}

\section{二维几何}

\subsection{二维几何基本操作}
\inputminted{cpp}{merge/Geo2D.cpp}

\subsection{$n\log n$ 半平面交}
\inputminted{cpp}{merge/HalfPlaneIntersection.cpp}

\subsection{三角形的心}
\inputminted{cpp}{improve/Triangle.cpp}

\subsection{圆与多边形面积交}
\inputminted{cpp}{merge/areaCT.cpp}

\subsection{圆的面积模板 ($n^2\log n$)}
\inputminted{cpp}{merge/CircleArea.cpp}

\subsection{凸包快速询问}
\inputminted{cpp}{improve/PlayWithConvex.cpp}

\subsection{Delaunay 三角剖分}
\inputminted{cpp}{improve/DelaunayTriangulation.cpp}

\section{三维几何}

\subsection{三维几何基本操作}
\inputminted{cpp}{merge/3DGeo.cpp}

\subsection{三维凸包求重心}
\inputminted{cpp}{src/三维凸包.cpp}

\subsection{求四点外界球}
\inputminted{cpp}{src/最小覆盖球.cpp}

\section{图论}

\subsection{Hungarian}
\inputminted{cpp}{improve/Hungarian.cpp}

\subsection{Hopcroft}
\inputminted{cpp}{src/Hopcroft.cpp}

\subsection{最大团}
\inputminted{cpp}{improve/MaximumClique.cpp}

\subsection{最小树形图}
\inputminted{cpp}{improve/LiuZhu.cpp}

\subsection{带花树}
\inputminted{cpp}{src/带花树.cpp}

\subsection{Dominator Tree}
\inputminted{cpp}{improve/DominatorTree.cpp}

\subsection{主流}
\inputminted{cpp}{improve/MincostFlow.cpp}

\subsection{无向图最小割}
\inputminted{cpp}{src/无向图最小割.cpp}

\section{数论}

\subsection{素数判定}
\inputminted{cpp}{src/素数判定.cpp}

\subsection{启发式分解}
\inputminted{cpp}{src/启发式分解.cpp}

\subsection{直线下整点个数}
\inputminted{cpp}{src/直线下格点统计.cpp}

\subsection{二次剩余}
\inputminted{cpp}{src/二次剩余.cpp}

\subsection{Pell 方程}
\inputminted{cpp}{src/Pell方程.cpp}

\section{代数}

\subsection{FFT}
\inputminted{cpp}{improve/FFT.cpp}

\subsection{线性规划}
\inputminted{cpp}{src/线性规划.cpp}

\subsection{Schreier-Sims}
\inputminted{cpp}{improve/SchreierSims.cpp}

\section{字符串}

\subsection{后缀数组 ( 倍增 )}
\inputminted{cpp}{src/后缀数组(nlogn).cpp}

\subsection{后缀自动机}
\inputminted{cpp}{src/后缀自动机.cpp}

\subsection{Manacher/ 扩展 KMP}
\inputminted{cpp}{merge/Manacher.cpp}

\subsection{字符串最小表示}
\inputminted{cpp}{src/字符串最小表示.cpp}

\subsection{后缀树 (With Pop Front)}
\inputminted{cpp}{improve/SuffixTree2.cpp}

\section{数据结构}

\subsection{Splay Tree}
\inputminted{cpp}{src/Splay.cpp}

\subsection{Link Cut Tree}
\inputminted{cpp}{improve/LCT.cpp}

\subsection{轻重链剖分}
\inputminted{cpp}{src/轻重链剖分.cpp}

\section{综合}

\subsection{DancingLinks}
\inputminted{cpp}{src/DancingLinks.cpp}

\subsection{日期公式}
\inputminted{cpp}{improve/日期公式.cpp}

\subsection{环状最长公共子序列}
\inputminted{cpp}{improve/CycleLongest.cpp}

\subsection{经纬度球面距离}
\inputminted{cpp}{src/经纬度求球面最短距离.cpp}

\subsection{长方体表面两点最短距离}
\inputminted{cpp}{src/长方体表面两点最短距离.cpp}

\section{其他}

\subsection{简易积分表}
\begin{footnotesize}
\noindent
\mbox{\vbox to 11pt{  \hbox{$
\int \frac{1}{1+x^2}dx = \tan^{-1}x
$}  }}
\
\mbox{\vbox to 11pt{  \hbox{$
\int \frac{1}{a^2+x^2}dx = \frac{1}{a}\tan^{-1}\frac{x}{a}
$}  }}
\\
\mbox{\vbox to 11pt{  \hbox{$
\int \frac{x}{a^2+x^2}dx = \frac{1}{2}\ln|a^2+x^2|
$}  }}
\
\mbox{\vbox to 11pt{  \hbox{$
\int \frac{x^2}{a^2+x^2}dx = x-a\tan^{-1}\frac{x}{a}
$}  }}
\\
\mbox{\vbox to 11pt{  \hbox{$
\int\sqrt{x^2 \pm a^2} dx  = \frac{1}{2}x\sqrt{x^2\pm a^2} 
%\nonumber \\ 
\pm\frac{1}{2}a^2 \ln \left | x + \sqrt{x^2\pm a^2} \right | 
$}  }}
\\
\mbox{\vbox to 11pt{  \hbox{$
\int  \sqrt{a^2 - x^2} dx  = \frac{1}{2} x \sqrt{a^2-x^2} 
%\nonumber \\  
+\frac{1}{2}a^2\tan^{-1}\frac{x}{\sqrt{a^2-x^2}}
$}  }}
\\
\mbox{\vbox to 11pt{  \hbox{$
\int \frac{x^2}{\sqrt{x^2 \pm a^2}} dx  = \frac{1}{2}x\sqrt{x^2 \pm a^2}
%\nonumber \\  
\mp \frac{1}{2}a^2 \ln \left| x + \sqrt{x^2\pm a^2} \right | 
$}  }}
\\
\mbox{\vbox to 11pt{  \hbox{$
\int \frac{1}{\sqrt{x^2 \pm a^2}} dx = \ln \left | x + \sqrt{x^2 \pm a^2} \right | 
$}  }}
\\
\mbox{\vbox to 11pt{  \hbox{$
\int \frac{1}{\sqrt{a^2 - x^2}} dx = \sin^{-1}\frac{x}{a} 
$}  }}
\
\mbox{\vbox to 11pt{  \hbox{$
\int \frac{x}{\sqrt{x^2\pm a^2}}dx = \sqrt{x^2 \pm a^2} 
$}  }}
\
\mbox{\vbox to 11pt{  \hbox{$
\int \frac{x}{\sqrt{a^2-x^2}}dx = -\sqrt{a^2-x^2} 
$}  }}
\\
\mbox{\vbox to 11pt{  \hbox{$
\int  \sqrt{a x^2 + b x + c} dx = 
\frac{b+2ax}{4a}\sqrt{ax^2+bx+c}
\nonumber \\  
+
\frac{4ac-b^2}{8a^{3/2}}\ln \left| 2ax + b + 2\sqrt{a(ax^2+bx^+c)}\right |
$}  }}
\\
\mbox{\vbox to 11pt{  \hbox{$
\int x^n e^{ax}\hspace{1pt}\text{d}x = \dfrac{x^n e^{ax}}{a} - 
\dfrac{n}{a}\int x^{n-1}e^{ax}\hspace{1pt}\text{d}x
$}  }} 
\\
\mbox{\vbox to 11pt{  \hbox{$
\int \sin^2 ax dx = \frac{x}{2} - \frac{1} {4a} \sin{2ax}
$}  }}
\
\mbox{\vbox to 11pt{  \hbox{$
\int \sin^3 ax dx = -\frac{3 \cos ax}{4a} + \frac{\cos 3ax} {12a} 
$}  }}
\\
\mbox{\vbox to 11pt{  \hbox{$
\int \cos^2 ax dx = \frac{x}{2}+\frac{ \sin 2ax}{4a} 
$}  }}
\
\mbox{\vbox to 11pt{  \hbox{$
\int \cos^3 ax dx = \frac{3 \sin ax}{4a}+\frac{ \sin 3ax}{12a} 
$}  }}
\\
\mbox{\vbox to 11pt{  \hbox{$
\int \tan ax dx = -\frac{1}{a} \ln \cos ax 
$}  }}
\
\mbox{\vbox to 11pt{  \hbox{$
\int \tan^2 ax dx = -x + \frac{1}{a} \tan ax 
$}  }}
\\
\mbox{\vbox to 11pt{  \hbox{$
\int x \cos ax dx = \frac{1}{a^2} \cos ax + \frac{x}{a} \sin ax 
$}  }}
\
\mbox{\vbox to 11pt{  \hbox{$
\int x^2 \cos ax dx = \frac{2 x \cos ax }{a^2} + \frac{ a^2 x^2 - 2  }{a^3} \sin ax 
$}  }}
\\
\mbox{\vbox to 11pt{  \hbox{$
\int x \sin ax dx = -\frac{x \cos ax}{a} + \frac{\sin ax}{a^2} 
$}  }}
\
\mbox{\vbox to 11pt{  \hbox{$
\int x^2 \sin ax dx =\frac{2-a^2x^2}{a^3}\cos ax +\frac{ 2 x \sin ax}{a^2} 
$}  }}
\end{footnotesize}


\subsection{常用结论}
\subsection{弦图}
设 $next(v)$ 表示 $N(v)$ 中最前的点 . 
令 $w*$ 表示所有满足 $A \in B$ 的 $w$ 中最后的一个点 , 
判断 $v \cup N(v)$ 是否为极大团 , 
只需判断是否存在一个 $w \in w*$, 
满足 $Next(w)=v$ 且 $|N(v)| + 1 \leq |N(w)|$ 即可 . 
\subsection{五边形数}
$
    \prod_{n=1}^{\infty}{(1-x^{n})}=\sum_{n=0}^{\infty}{(-1)^{n}(1-x^{2n+1})x^{n(3n+1)/2}}
$
\subsection{重心}
半径为 $r$ , 圆心角为 $\theta$ 的扇形重心与圆心的距离为 $\frac{4r\sin(\theta/2)}{3\theta}$ \\
半径为 $r$ , 圆心角为 $\theta$ 的圆弧重心与圆心的距离为 $\frac{4r\sin^3(\theta/2)}{3(\theta-\sin(\theta))}$ \\
\subsection{第二类 Bernoulli number}
\begin{align*}
    B_m &= 1 - \sum_{k=0}^{m-1}{\binom{m}{k}\frac{B_{k}}{m-k+1}} \\
    S_m(n) &= \sum_{k=1}^{n}{k^{m}} = \frac{1}{m+1}\sum_{k=0}^{m}{\binom{m+1}{k}B_{k}n^{m+1-k}}
\end{align*}
\subsection{Stirling 数}
第一类 :n 个元素的项目分作 k 个环排列的方法数目\\
\begin{align*}
    s(n, k) &= (-1)^{n+k}|s(n, k)| \\
    |s(n, 0)| &=0\\ 
    |s(1, 1)| &=1 \\
    |s(n, k)| &=|s(n-1, k-1)|+(n-1)*|s(n-1, k)|
\end{align*}
第二类 :n 个元素的集定义 k 个等价类的方法数\\
\begin{align*}
    S(n,1)&=S(n,n)=1\\
    S(n,k)&=S(n-1,k-1)+k*S(n-1,k)
\end{align*}
\subsection{三角公式}

\begin{footnotesize}
\noindent
\mbox{\vbox to 11pt{  \hbox{$
\sin(a \pm b) = \sin a \cos b \pm \cos a \sin b
$}  }}
\
\mbox{\vbox to 11pt{  \hbox{$
\cos(a \pm b) = \cos a \cos b \mp \sin a \sin b
$}  }}
\\
\mbox{\vbox to 11pt{  \hbox{$
\tan(a \pm b) = \frac{\tan(a)\pm\tan(b)}{1 \mp \tan(a)\tan(b)}
$}  }}
\
\mbox{\vbox to 11pt{  \hbox{$
\tan(a) \pm \tan(b) = \frac{\sin(a \pm b)}{\cos(a)\cos(b)}
$}  }}
\\
\mbox{\vbox to 11pt{  \hbox{$
\sin(a) + \sin(b) = 2\sin(\frac{a + b}{2})\cos(\frac{a - b}{2})
$}  }}
\
\mbox{\vbox to 11pt{  \hbox{$
\sin(a) - \sin(b) = 2\cos(\frac{a + b}{2})\sin(\frac{a - b}{2})
$}  }}
\\
\mbox{\vbox to 11pt{  \hbox{$
\cos(a) + \cos(b) = 2\cos(\frac{a + b}{2})\cos(\frac{a - b}{2})
$}  }}
\
\mbox{\vbox to 11pt{  \hbox{$
\cos(a) - \cos(b) = -2\sin(\frac{a + b}{2})\sin(\frac{a - b}{2})
$}  }}
\\
\mbox{\vbox to 11pt{  \hbox{$
\sin(na) = n\cos^{n-1}a\sin a - \binom{n}{3}\cos^{n-3}a \sin^3a + \binom{n}{5}\cos^{n-5}a\sin^5a - \dots
$}  }}
\\
\mbox{\vbox to 11pt{  \hbox{$
\cos(na) = \cos^{n}a - \binom{n}{2}\cos^{n-2}a \sin^2a + \binom{n}{4}\cos^{n-4}a\sin^4a - \dots
$}  }}

\end{footnotesize}


\subsection{Java 读入优化}
\inputminted{java}{src/Main.java}

\subsection{Vimrc}
\inputminted{text}{src/vimrc.vim}

\end{spacing}
\end{multicols}

\newpage
%l2LaTeX from sheet 'Sheet1'
\begin{table}
\centering
\begin{tabular}{|r|r|r|r|r|r|r|}
	\hline
	  n & log10 n & n! & n C(n/2) & LCM(1...n) & Pn & Bn \\
	\hline
	  2 & 0.30 & 2 & 2 & 2 & 2 & 2 \\
	\hline
	  3 & 0.48 & 6 & 3 & 6 & 3 & 5 \\
	\hline
	  4 & 0.60 & 24 & 6 & 12 & 5 & 15 \\
	\hline
	  5 & 0.70 & 120 & 10 & 60 & 7 & 52 \\
	\hline
	  6 & 0.78 & 720 & 20 & 60 & 11 & 203 \\
	\hline
	  7 & 0.85 & 5040 & 35 & 420 & 15 & 877 \\
	\hline
	  8 & 0.90 & 40320 & 70 & 840 & 22 & 4140 \\
	\hline
	  9 & 0.95 & 362880 & 126 & 2520 & 30 & 21147 \\
	\hline
	  10 & 1 & 3628800 & 252 & 2520 & 42 & 115975 \\
	\hline
	  11 &   & 39916800 & 462 & 27720 & 56 & 678570 \\
	\hline
	  12 &   & 479001600 & 924 & 27720 & 77 & 4213597 \\
	\hline
	  15 &   &   & 6435 & 360360 & 176 & 1382958545 \\
	\hline
	  20 &   &   & 184756 & 232792560 & 627 &  \\
	\hline
	  25 &   &   & 5200300 &   & 1958 &  \\
	\hline
	  30 &   &   & 155117520 &   & 5604 &  \\
	\hline
	  40 &   &   &   &   & 37338 &  \\
	\hline
	  50 &   &   &   &   & 204226 &  \\
	\hline
	  70 &   &   &   &   & 4087968 &  \\
	\hline
	  100 &   &   &   &   & 190569292 &  \\
	\hline
	\end{tabular}
\end{table}





\end{document}
%THE SCL ENDS
