\begin{lstlisting}
模板需要配合miller\_rabin一起使用.
调用factor()函数, 会返回vector<long long>, 表示分解结果. (例如分解12, 会返回2, 2和3)
namespace pollard_rho {
	//可以改成LL*LL%LL的形式
	inline long long multiply(const long long &x, const long long &y, const long long &p) {
		long long ans = 0, num = x % p;
		for (long long i = y; i > 0; i >>= 1) {
			if (i & 1) {
				(ans += num) %= p;
			}
			(num += num) %= p;
		}
		return ans;
	}
 
	inline long long gcd(long long x, long long y) {
		while (y > 0) {
			x %= y;
			swap(x, y);
		}
		return x;
	}
 
	inline long long pollard_rho(const long long &n, const long long &c) {
		long long x = rand() % (n - 1) + 1, y = x;
		int head = 1, tail = 2;
		while (true) {
			x = multiply(x, x, n);
			if ((x += c) >= n) {
				x -= n;
			}
			if (x == y) {
				return n;
			}
			long long d = gcd(abs(x - y), n);
			if (d > 1 && d < n) {
				return d;
			}
			if ((++head) == tail) {
				y = x;
				tail <<= 1;
			}
		}
	}
 
	inline vector<long long> mergy(const vector<long long> &a, const vector<long long> &b) {
		vector<long long> vec;
		for (int i = 0; i < (int)a.size(); ++i) {
			vec.push_back(a[i]);
		}
		for (int i = 0; i < (int)b.size(); ++i) {
			vec.push_back(b[i]);
		}
		return vec;
	}
 
	inline vector<long long> factor(const long long &n) {
		if (n <= 1) {
			return vector<long long>();
		}
		if (miller_rabin::prime(n)) {
			return vector<long long>(1, n);
		}
		long long p = n;
		while (p >= n) {
			p = pollard_rho(n, rand() % (n - 1) + 1);
		}
		return mergy(factor(n / p), factor(p));
	}
}
\end{lstlisting}
