牛顿迭代
x1=x0-func(x0)/func1(x0);//进行牛顿迭代计算
我们要求 f(x)=0 的解。func(x)为原方程,func1 为原方程的导数方程
图同构 hash
F t (i) = (F t−1 (i) × A + ∑ F t−1 (j) × B + ∑ F t−1 (j) × C + D × (i == a)) mod P
i→j
j→i
枚举点 a,迭代 K 次后求得的Fk (a)就是 a 点所对应的 hash 值。
其中 K、A、B、C、D、P 为 hash 参数,可自选。


设正整数 n 的质因数分解为 n = ∏pi^ai,则 x^2+y^2=n 有整数解的充要条件是 n 中不存在形
如 pi≡3(mod 4) &(and) 指数 ai 为奇数的质因数 pi

Pick 定理:简单多边形,不自交。(严格在多边形内部的整点数*2 +在边上的整点数– 2)/2 =面积

定理 1:最小覆盖数 = 最大匹配数
定理 2:最大独立集 S 与 最小覆盖集 T 互补。
算法:
1. 做最大匹配,没有匹配的空闲点∈S
2. 如果 u∈S 那么 u 的临点必然属于 T
3. 如果一对匹配的点中有一个属于 T 那么另外一个属于 S
4. 还不能确定的,把左子图的放入 S,右子图放入 T
算法结束

p 是素数且 2^p-1 的是素数,n 不超过 258 的全部梅森素数终于确定!是
n=2,3,5,7,13,17,19,31,61,89,107,127

有上下界网络流,求可行流部分,增广的流量不是实际流量。若要求实际流量应该强算一遍源点出去的流量。
求最小下届网络流:
方法一:加 t-s 的无穷大流,求可行流,然后把边反向后(减去下届网络流),在残留网络中从汇到源做最大流。
方法二:在求可行流的时候,不加从汇到源的无穷大边,得到最大流 X, 加上从汇到源无穷大边后,再求最大流得到 Y。那么 Y 即是答案最小下界网络流。
原因:感觉上是在第一遍已经把内部都消耗光了,第二遍是必须的流量。

平面图一定存在一个度小于等于 5 的点,且可以四染色
( 欧拉公式 ) 设 G 是连通的平面图,n , m, r 分别是其顶点数、边数和面数,n-m+r=2
极大平面图 m≤3n-6

Fibonacci
gcd(2^(a)-1,2^(b)-1)=(2^gcd(a,b))-1.
gcd(F[n],F[m])=F[gcd(n,m)] (牛书,P228)
Fibonacci 质数(和前面所有的 Fibonacci 数互质)
(大多已经是质数了,可能有 BUG 吧,不确定)
定理:如果 a 是 b 的倍数,那么 Fa 是 Fb 的倍数。

二次剩余
p 为奇素数,若(a,p)=1, a 为 p 的二次剩余必要充分条件为 a^((p-1)/2)mod p=1.(否则为 p-1)
p 为奇素数, = a(mod p),a 为 p 的 b 次剩余的必要充分条件为若 a^((p-1)/ (p-1 和 b的最大公约数)) mod p=1.


平方数的和是平方数的问题。
a[0] := 0;
s := 0;
for i := 1 to n - 2 do
begin
	a[i] := a[i - 1] + 1;
	s := s + sqr(a[i]);
end;
{======s + sqr(a[n-1]) + sqr(a[n]) = k^2=======}
a[n - 1] := a[n - 2];
repeat
	a[n - 1] := a[n - 1] + 1;
until odd(s + sqr(a[n - 1])) and (a[n - 1] > 2);
a[n] := (s + sqr(a[n - 1]) - 1) shr 1;
知道 s 和 a[n-1]后,直接求了 a[n].神奇了点。

其实。有当 n 为奇数:n^2 + ((n^2 - 1) div 2)^2 = ((n^2 + 1) div 2)^2
所以有 3 4-- 5 12 -- 7 24 -- 9 40 -- 11 60 ....
a=k*(s^2 - t^2);
b=2*k*s*t
c=k(s^2 + t^2);
则 c^2=a^2+b^2 完全的公式
