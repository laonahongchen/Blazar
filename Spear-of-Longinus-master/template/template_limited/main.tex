%!TEX program = xelatex
\documentclass[landscape, twoside, a4paper]{article}


%\usepackage{ctex}
\usepackage{listings, color, fontspec, minted, setspace, titlesec, fancyhdr, dingbat, mdframed, multicol}
\usepackage{graphicx, amssymb, amsmath, textcomp, booktabs, xeCJK}
\usepackage[left=1cm,right=1cm,top=1.3cm,bottom=0.5cm,columnsep=1cm,dvipdfm]{geometry}
\allowdisplaybreaks
%\setlength{\parindent}{0em}
%\setlength{\mathindent}{0pt}
\lstset{breaklines}
\let\cleardoublepage\relax
\titleformat{\section}{\large}{}{0pt}{}
\titleformat{\subsection}{\normalsize}{}{0pt}{}
\titleformat{\subsubsection}{\small}{}{0pt}{}
%\titleformat{\subsection}{\normalfont\normalsize\bfseries}{\thesubsection}{1em}{}
%\titleformat{\section}{\normalfont\large\bfseries}{Chapter \,\thesection}{10pt}{\large}
%\titleformat{\subsection}{\normalfont\normalsize\bfseries}{\thesubsection}{1em}{}
%\titleformat{\subsubsection}{\normalfont\small\bfseries}{\thesubsubsection}{1em}{}
%\titleformat{\subsubsection}{\normalfont\footnotesize\bfseries}{\thesubsubsubsection}{1em}{}
\titlespacing*{\section} {0pt}{0pt}{0pt}
\titlespacing*{\subsection} {0pt}{0pt}{0pt}
\titlespacing*{\subsubsection} {0pt}{-1pt}{-1pt}
%\titlespacing*{\subsubsection}{0pt}{-1pt}{-1pt}
%configure fonts
%\setmonofont{Menlo}[Scale=0.8]
%\setCJKmainfont{STHeiti}

\renewcommand{\theFancyVerbLine}{\sffamily \textcolor[rgb]{0.5,0.5,0.5}{\scriptsize {\arabic{FancyVerbLine}}}}

\setminted[cpp]{
	style=xcode,
	mathescape,
	linenos,
	autogobble,
	baselinestretch=0.5,
	tabsize=4,
	fontsize=\small,
	%bgcolor=Gray,
	frame=single,
	framesep=1mm,
	framerule=0.3pt,
	numbersep=1mm,
	breaklines=true,
	breaksymbolsepleft=2pt,
	%breaksymbolleft=\raisebox{0.8ex}{ \small\reflectbox{\carriagereturn}}, %not moe!
	%breaksymbolright=\small\carriagereturn,
	breakbytoken=false,
}
\setminted[java]{
	style=xcode,
	mathescape,
	linenos,
	autogobble,
	baselinestretch=1.0,
	tabsize=4,
	%bgcolor=Gray,
	frame=single,
	framesep=1mm,
	framerule=0.3pt,
	numbersep=1mm,
	breaklines=true,
	breaksymbolsepleft=2pt,
	%breaksymbolleft=\raisebox{0.8ex}{ \small\reflectbox{\carriagereturn}}, %not moe!
	%breaksymbolright=\small\carriagereturn,
	breakbytoken=false,
}
\setminted[text]{
	style=xcode,
	mathescape,
	linenos,
	autogobble,
	baselinestretch=1.0,
	tabsize=4,
	%bgcolor=Gray,
	frame=single,
	framesep=1mm,
	framerule=0.3pt,
	numbersep=1mm,
	breaklines=true,
	breaksymbolsepleft=2pt,
	%breaksymbolleft=\raisebox{0.8ex}{ \small\reflectbox{\carriagereturn}}, %not moe!
	%breaksymbolright=\small\carriagereturn,
	breakbytoken=false,
}

\pagestyle{fancy}
\setlength{\headsep}{0.1cm}
\rhead{Page \thepage}
\lhead{上海交通大学 Shanghai Jiao Tong University}
 %页眉线宽,设为0可以去页眉线
%\setlength{\skip\footins}{0.5cm}    %脚注与正文的距离           
%\renewcommand{\footnotesize}{}      %设置脚注字体大小           
%\renewcommand{\footrulewidth}{1pt}  %脚注线的宽度           

\newcommand{\cppcode}[1]{
	\inputminted[mathescape,
	frame=lines,linenos]{cpp}{source/#1}
}
\newcommand{\javacode}[1]{
	\inputminted[mathescape,
	frame=lines,linenos]{java}{source/#1}
}
\begin{document}\scriptsize
	%\title{}
	%\author{\emph{Shanghai Jiao Tong University}}
%	\date{\today}
%	\maketitle
	
	\begin{center}
		\textbf{\LARGE{Spear of Longinus}}
		
		
		\textbf{\small{Shanghai Jiao Tong University}}
	\end{center}
	
	%\clearpage
	\begin{multicols}{2}
		
		\begin{spacing}{0}
			\tableofcontents
		\end{spacing}
		\clearpage
		
		\begin{spacing}{0.8}
			

\section{数论}


%\subsection{快速求逆元}
%返回结果:$$x^{-1}(mod)$$
%\indent 使用条件:$x \in [0, mod)$并且$x$与$mod$互质
%\cppcode{number-theory/inverse.cpp}

\subsection{扩展欧几里德算法}


\cppcode{number-theory/extended-euclid.cpp}


\subsection{中国剩余定理}

\cppcode{number-theory/chinese-remainder-theorem.cpp}


\subsection{中国剩余定理2}

\cppcode{number-theory/china2.cpp}


%\subsection{组合数取模}

%\cppcode{number-theory/CnmmodP.cpp}


\subsection{扩展小步大步}

\cppcode{number-theory/exBSGS.cpp}


\subsection{卢卡斯定理}

\cppcode{number-theory/Lucas.cpp}



\subsection{Miller Rabin 素数测试}


\cppcode{number-theory/miller-rabin.cpp}


\subsection{Pollard Rho 大数分解}


时间复杂度:$\mathcal{O}(n^{1/4})$


\cppcode{number-theory/pollard-rho.cpp}




\subsection{快速数论变换(zky)}

返回结果:$$c_i=\sum_{0 \leq j \leq i} a_j \cdot b_{i-j} (mod) \ (0 \leq i < n)$$

\indent 使用说明:$magic$是$mod$的原根\\

\indent 时间复杂度:$\mathcal{O}(n log n)$

\cppcode{number-theory/number-theoretic-transform.cpp}


%\subsection{快速数论变换(lyx)}

%\cppcode{number-theory/DFT.cpp}


\subsection{原根}

\cppcode{number-theory/primeroot.cpp}


\subsection{线性递推}

\cppcode{number-theory/linear-recurrence.cpp}


%\subsection{线性筛}

%\cppcode{number-theory/linear-sieve.cpp}


%\subsection{离散对数}


%\subsection{离散平方根}


%\subsection{佩尔方程求解}


\subsection{直线下整点个数}


返回结果:$$\sum_{0 \leq i < n} \lfloor \frac{a + b \cdot i}{m} \rfloor$$
\cppcode{number-theory/lattice-count.cpp}


\section{数值}


\subsection{高斯消元}

\cppcode{numerical-algorithm/Gauss.cpp}


\subsection{快速傅立叶变换}


返回结果:$$c_i=\sum_{0 \leq j \leq i} a_j \cdot b_{i-j} \ (0 \leq i < n)$$

\indent 时间复杂度:$\mathcal{O}(n log n)$


\cppcode{numerical-algorithm/fast-fourier-transform.cpp}


\subsection{单纯形法求解线性规划}


返回结果:$$max\{c_{1 \times m} \cdot x_{m \times 1} \ | \ x_{m \times 1} \geq 0_{m \times 1}, a_{n \times m} \cdot x_{m \times 1} \leq b_{n \times 1}\}$$


\cppcode{numerical-algorithm/linear-programming-simplex.cpp}


\subsection{自适应辛普森}


\cppcode{numerical-algorithm/adaptive-simpson.cpp}



\subsection{多项式求根}


\cppcode{numerical-algorithm/polyroot.cpp}


%\subsection{牛顿迭代法}


%\subsection{多项式方程求解}


%\subsection{最小二乘法}


\section{数据结构}


\subsection{平衡的二叉查找树}


\subsubsection{Treap}


\cppcode{data-structure/Treap.cpp}


%\subsubsection{Splay}


%\cppcode{data-structure/Splay.cpp}


\subsection{坚固的数据结构}


%\subsubsection{坚固的线段树}


%\cppcode{data-structure/persistent-segment-tree.cpp}


\subsubsection{坚固的平衡树}


\cppcode{data-structure/fhqTreap.cpp}


\subsubsection{坚固的字符串}


%\begin{enumerate}

%\item ext库中的rope

%\cppcode{data-structure/crope.cpp}

%\item 可持久化平衡树实现的rope

%\cppcode{data-structure/rope.cpp}

%\end{enumerate}


\subsubsection{坚固的左偏树}

\cppcode{data-structure/lefttree.cpp}


%\subsubsection{不坚固的斜堆}

%\cppcode{data-structure/SkewHeap.cpp}


\subsection{树上的魔术师}


%\subsubsection{轻重树链剖分(zky)}

%\cppcode{data-structure/HLD.cpp}


%\subsubsection{轻重树链剖分(lyx)}

%\cppcode{data-structure/tree-chain.cpp}


\subsubsection{Link Cut Tree(zky)}

\cppcode{data-structure/LCT.cpp}


\subsubsection{Link Cut Tree(Splay)}

\cppcode{data-structure/Link-cut-tree.cpp}


%\subsubsection{AAA Tree}

%\cppcode{data-structure/toptree.cpp}


%\subsection{ST}

%\cppcode{data-structure/Rmq.cpp}


\subsection{可持久化线段树}

\cppcode{data-structure/ChairTree.cpp}


%\subsection{可持久化Trie}

%\cppcode{data-structure/ChairTrie.cpp}


\subsection{k-d树}

\cppcode{data-structure/kd-tree.cpp}



\subsection{莫队算法}

\cppcode{data-structure/mo-team.cpp}

%\subsection{树上在线莫队}

%\cppcode{data-structure/mo-teamontree.cpp}


%\subsection{整体二分}

%\cppcode{data-structure/binary-search.cpp}


\subsection{树状数组kth}

\cppcode{data-structure/fenwicktree.cpp}


\subsection{虚树}

\cppcode{data-structure/virtualtree.cpp}


\subsection{点分治(zky)}

\cppcode{data-structure/pointdivide.cpp}


\subsection{元芳树}

\cppcode{data-structure/yuanfang.cpp}



\section{图论}


%\subsection{强连通分量}


%\cppcode{graph-theory/strongly-connected-components.cpp}


%\subsection{双连通分量}


\subsection{点双连通分量(lyx)}

\cppcode{graph-theory/pointtwo.cpp}


%\subsubsection{边双连通分量}


\subsection{2-SAT问题(强连通分量)}


\cppcode{graph-theory/two-satisfiability.cpp}


\subsection{二分图最大匹配}


\subsubsection{Hungary算法}


时间复杂度:$\mathcal{O}(V \cdot E)$


\cppcode{graph-theory/Hungarian.cpp}


\subsubsection{Hopcroft Karp算法}


时间复杂度:$\mathcal{O}(\sqrt{V} \cdot E)$


\cppcode{graph-theory/HK.cpp}


\subsection{二分图最大权匹配}


时间复杂度:$\mathcal{O}(V^4)$


\cppcode{graph-theory/maximum-weight-matching.cpp}


\subsection{最大流(dinic)}

时间复杂度:$\mathcal{O}(V^2 \cdot E)$

\cppcode{graph-theory/dinic.cpp}



\subsection{最大流(sap)}

时间复杂度:$\mathcal{O}(V^2 \cdot E)$

\cppcode{graph-theory/Sap.cpp}



\subsection{上下界网络流}


$B(u,v)$表示边$(u,v)$流量的下界,$C(u,v)$表示边$(u,v)$流量的上界,$F(u,v)$表示边$(u,v)$的流量。设$G(u,v) = F(u,v) - B(u,v)$,显然有

$$0 \leq G(u,v) \leq C(u,v)-B(u,v)$$


\subsubsection{无源汇的上下界可行流}


建立超级源点$S^*$和超级汇点$T^*$,对于原图每条边$(u,v)$在新网络中连如下三条边:$S^* \rightarrow v$,容量为$B(u,v)$;$u \rightarrow T^*$,容量为$B(u,v)$;$u \rightarrow v$,容量为$C(u,v) - B(u,v)$。最后求新网络的最大流,判断从超级源点$S^*$出发的边是否都满流即可,边$(u,v)$的最终解中的实际流量为$G(u,v)+B(u,v)$。


\subsubsection{有源汇的上下界可行流}


从汇点$T$到源点$S$连一条上界为$\infty$,下界为$0$的边。按照\textbf{无源汇的上下界可行流}一样做即可,流量即为$T \rightarrow S$边上的流量。


\subsubsection{有源汇的上下界最大流}


\begin{enumerate}

\item 在\textbf{有源汇的上下界可行流}中,从汇点$T$到源点$S$的边改为连一条上界为$\infty$,下届为$x$的边。$x$满足二分性质,找到最大的$x$使得新网络存在\textbf{无源汇的上下界可行流}即为原图的最大流。

\item 从汇点$T$到源点$S$连一条上界为$\infty$,下界为$0$的边,变成无源汇的网络。按照\textbf{无源汇的上下界可行流}的方法,建立超级源点$S^*$和超级汇点$T^*$,求一遍$S^* \rightarrow T^*$的最大流,再将从汇点$T$到源点$S$的这条边拆掉,求一次$S \rightarrow T$的最大流即可。

\end{enumerate}


\subsubsection{有源汇的上下界最小流}


\begin{enumerate}

\item 在\textbf{有源汇的上下界可行流}中,从汇点$T$到源点$S$的边改为连一条上界为$x$,下界为$0$的边。$x$满足二分性质,找到最小的$x$使得新网络存在\textbf{无源汇的上下界可行流}即为原图的最小流。

\item 按照\textbf{无源汇的上下界可行流}的方法,建立超级源点$S^*$与超级汇点$T^*$,求一遍$S^* \rightarrow T^*$的最大流,但是注意这一次不加上汇点$T$到源点$S$的这条边,即不使之改为无源汇的网络去求解。求完后,再加上那条汇点$T$到源点$S$上界$\infty$的边。因为这条边下界为$0$,所以$S^*$,$T^*$无影响,再直接求一次$S^* \rightarrow T^*$的最大流。若超级源点$S^*$出发的边全部满流,则$T \rightarrow S$边上的流量即为原图的最小流,否则无解。

\end{enumerate}


\subsection{最小费用最大流}


\subsubsection{稀疏图}


时间复杂度:$\mathcal{O}(V \cdot E^2)$


\cppcode{graph-theory/minimum-cost-flow-spfa.cpp}


\subsubsection{稠密图}


使用条件:费用非负\\

\indent 时间复杂度:$\mathcal{O}(V \cdot E^2)$


\cppcode{graph-theory/minimum-cost-flow-zkw.cpp}


\subsection{一般图最大匹配}


时间复杂度:$\mathcal{O}(V^3)$


\cppcode{graph-theory/maximum-matching-blossom.cpp}


\subsection{无向图全局最小割}


时间复杂度:$\mathcal{O}(V^3)$\\

\indent 注意事项:处理重边时,应该对边权累加


\cppcode{graph-theory/minimum-cut-stoer-wagner.cpp}


%\subsection{最小树形图}


%\subsection{有根树的同构}


%时间复杂度:$\mathcal{O}(V log V)$


%\cppcode{graph-theory/rooted-tree-isomorphism.cpp}


%\subsection{度限制生成树}


%\subsection{弦图相关}


%\subsubsection{弦图的判定}


%\subsubsection{弦图的团数}


\subsection{哈密尔顿回路(ORE性质的图)}


ORE性质:$$\forall x,y \in V \wedge (x,y) \notin E \ \ s.t. \ \ deg_x+deg_y \geq n$$

\indent 返回结果:从顶点$1$出发的一个哈密尔顿回路\\

\indent 使用条件:$n \geq 3$


\cppcode{graph-theory/hamiltonian-circuit-ore.cpp}


\subsection{必经点树}

\cppcode{graph-theory/dominator-tree.cpp}


\section{字符串}


\subsection{模式匹配}


\subsubsection{KMP算法}


\cppcode{string-manipulation/KMP.cpp}


\subsubsection{扩展KMP算法}


返回结果:$$next_i = lcp(text, text_{i \dots n-1})$$


\cppcode{string-manipulation/ExtKMP.cpp}


\subsubsection{AC自动机}


\cppcode{string-manipulation/ACmachine.cpp}


\subsection{后缀三姐妹}


\subsubsection{后缀数组}


\cppcode{string-manipulation/Sa.cpp}


\subsubsection{后缀数组(dc3)}

\cppcode{string-manipulation/DC3.cpp}


\subsubsection{后缀自动机}
\indent 多串LCS
\indent 对一个串建后缀自动机,其他串在上面匹配,因为是求所有串的公共子串,所以每个点记录每个串最长匹配长度的最小值,最后找到所有点中最长的一个即可。一个注意事项就是,当走到一个点时,还要更新它的parent树上的祖先的匹配长度,数组开两倍啦啦啦!


\indent 各长度字串出现次数最大值
\indent 给一个字符串S,令F(x)表示S的所有长度为x的子串中,出现次数的最大值。
\indent 构建字符串的自动机,对于每个节点,right集合大小就是出现次数,maxs就是它代表的最长长度,那么我们用|right(x)|去更新f[maxs[x]]的值,最后从大到小用f[i]去更新f[i-1]的值即可

\cppcode{string-manipulation/Sam-LCS.cpp}

%\subsubsection{后缀自动机-各长度字串出现次数最大值}

%\cppcode{string-manipulation/Sam-max.cpp}

%\subsubsection{后缀自动机-两串LCS}

%\cppcode{string-manipulation/Sam-2LCS.cpp}


\subsection{回文三兄弟}

\subsubsection{马拉车}

\cppcode{string-manipulation/Manacher.cpp}

%\subsubsection{回文树(lyx)}

%\cppcode{string-manipulation/PAMM.cpp}

\subsubsection{回文自动机(zky)}

\cppcode{string-manipulation/PAM.cpp}





\subsection{循环串最小表示}

\cppcode{string-manipulation/minexpress.cpp}


\section{计算几何}


\subsection{二维基础}


\subsubsection{点类}
\cppcode{computational-geometry/point.cpp}


\subsubsection{圆类}
\cppcode{computational-geometry/circle.cpp}

\subsubsection{凸包}


\cppcode{computational-geometry/convex-hull.cpp}


\subsubsection{半平面交}

\cppcode{computational-geometry/halfplaneintersection.cpp}


%\subsubsection{最近点对}

%\cppcode{computational-geometry/closest-pair-of-points.cpp}


\subsubsection{最小圆覆盖}

\cppcode{computational-geometry/mincir.cpp}


%\subsubsection{凸包快速询问}

%\cppcode{computational-geometry/PlayWithConvex.cpp}


%\subsection{三维基础}


%\subsubsection{点类}


%\subsubsection{凸包}


%\subsubsection{绕轴旋转}


\subsection{多边形}


\subsubsection{判断点在多边形内部}


\cppcode{computational-geometry/point-in-polygon.cpp}




%\subsubsection{旋转卡壳}


%\subsubsection{动态凸包}


%\subsubsection{点到凸包的切线}


%\subsubsection{直线与凸包的交点}


%\subsubsection{凸多边形的交集}


%\subsubsection{凸多边形内的最大圆}


%\subsection{圆}


%\subsubsection{圆类}


%\subsubsection{圆的交集}


%\subsubsection{最小覆盖圆}


%\subsubsection{最小覆盖球}


%\subsubsection{判断圆存在交集}


%\subsubsection{圆与多边形的交集}


%\subsection{三角形}


%\subsubsection{三角形的内心}


%\subsubsection{三角形的外心}


%\subsubsection{三角形的垂心}


%\subsection{黑暗科技}


%\subsubsection{平面图形的转动惯量}


%\subsubsection{平面区域处理}


%\subsubsection{Vonoroi图}


\section{其他}


\subsection{斯坦纳树}

\cppcode{miscellany/Steiner-Tree.cpp}


\subsection{无敌的读入优化}

\cppcode{miscellany/Reader.cpp}


\subsection{最小树形图}

\cppcode{miscellany/mintreegraph.cpp}


\subsection{DLX}

\cppcode{miscellany/DLX.cpp}


%\subsection{插头DP}

%\cppcode{miscellany/plugdp.cpp}



\subsection{某年某月某日是星期几}


\cppcode{miscellany/what-day-is-today.cpp}


\subsection{枚举大小为$k$的子集}


使用条件:$k > 0$


\cppcode{miscellany/subset-of-size-k.cpp}


\subsection{环状最长公共子串}


\cppcode{miscellany/cyclic-longest-common-string.cpp}


\subsection{LLMOD}


\cppcode{miscellany/LLMOD.cpp}


%\subsection{搜索}


%\subsubsection{Dancing Links X}


\section{Java}
\subsection{基础模板}
\javacode{template.java}



		\end{spacing}
	\end{multicols}
\end{document}
