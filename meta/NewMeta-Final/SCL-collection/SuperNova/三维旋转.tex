\begin{lstlisting}
//sgu265
#include<cstring>
#include<cstdio>
#include<cmath>
#include<algorithm>
using namespace std;
const double pi = acos(-1.0);
int n, m; char ch1; bool flag;
double a[4][4], s1, s2, x, y, z, w, b[4][4], c[4][4];
double sqr(double x)
{
	return x*x;
}
int main()
{
	scanf("%d\n", &n);
	memset(b, 0, sizeof(b));
	b[0][0] = b[1][1] = b[2][2] = b[3][3] = 1;//initial matrix
	for(int i = 1; i <= n; i++)
	{
		scanf("%c", &ch1);
		if(ch1 == 'T')
		{
			//plus each coordinate by a number (x, y, z)
			scanf("%lf %lf %lf\n", &x, &y, &z);
			memset(a, 0, sizeof(a));
			a[0][0] = 1; a[3][0] = x;
			a[1][1] = 1; a[3][1] = y;
			a[2][2] = 1; a[3][2] = z;
			a[3][3] = 1;
		}else if(ch1 == 'S')
		{
			//multiply each coordinate by a number (x, y, z)
			scanf("%lf %lf %lf\n", &x, &y, &z);
			memset(a, 0, sizeof(a));
			a[0][0] = x;
			a[1][1] = y;
			a[2][2] = z;
			a[3][3] = 1;
		}else
		{
			//rotate in a clockwise about the ray from the origin through (x, y, z);
			scanf("%lf %lf %lf %lf\n", &x, &y, &z, &w);
			w = w*pi/180;
			memset(a, 0, sizeof(a));
			s1 = x*x+y*y+z*z;
			a[3][3] = 1;
			a[0][0] = ((y*y+z*z)*cos(w)+x*x)/s1;			
			a[0][1] = x*y*(1-cos(w))/s1+z*sin(w)/sqrt(s1);	
			a[0][2] = x*z*(1-cos(w))/s1-y*sin(w)/sqrt(s1);
			a[1][0] = x*y*(1-cos(w))/s1-z*sin(w)/sqrt(s1);	
			a[1][1] = ((x*x+z*z)*cos(w)+y*y)/s1;			
			a[1][2] = y*z*(1-cos(w))/s1+x*sin(w)/sqrt(s1);
			a[2][0] = x*z*(1-cos(w))/s1+y*sin(w)/sqrt(s1);	
			a[2][1] = y*z*(1-cos(w))/s1-x*sin(w)/sqrt(s1);	
			a[2][2] = ((x*x+y*y)*cos(w)+z*z)/s1;
		}
		memset(c, 0, sizeof(c));
		for(int i = 0; i < 4; i++)
			for(int j = 0; j < 4; j++)
				for(int k = 0; k < 4; k++)
					c[i][j] += b[i][k]*a[k][j];
		memcpy(b, c, sizeof(c));
	}
	scanf("%d", &m);
	for(int i = 1; i <= m; i++)
	{
		scanf("%lf%lf%lf", &x, &y, &z);//initial vector
		printf("%lf %lf %lf\n", x*b[0][0]+y*b[1][0]+z*b[2][0]+b[3][0], x*b[0][1]+y*b[1][1]+z*b[2][1]+b[3][1], x*b[0][2]+y*b[1][2]+z*b[2][2]+b[3][2]);
	}
	return 0;
}
\end{lstlisting}
